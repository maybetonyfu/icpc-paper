\documentclass[conference]{IEEEtran}
\IEEEoverridecommandlockouts

\usepackage{cite}
\usepackage{amsmath,amssymb,amsfonts}
\usepackage{algorithmic}
\usepackage{graphicx}
\usepackage{textcomp}
\usepackage{xcolor}
\usepackage{tabularx}
\usepackage{minted}
\makeatletter
\let\@float@c@listing\@caption
\makeatother
\usepackage{graphicx}
\usepackage[hidelinks]{hyperref}

\usemintedstyle{borland}

\newcommand{\chameleon}{ChameleonIDE}
\newcommand{\pjs}[1]{\textcolor{blue}{\textsc{Pjs}: #1}}
\newcommand{\jw}[1]{\textcolor{orange}{\textsc{JW}: #1}}
\newcommand{\td}[1]{\textcolor{teal}{\textsc{td}: #1}}
\newcommand{\tf}[1]{\textcolor{magenta}{\textsc{tf}: #1}}

\makeatletter % changes the catcode of @ to 11
\newcommand{\linebreakand}{%
  \end{@IEEEauthorhalign}
  \hfill\mbox{}\par
  \mbox{}\hfill\begin{@IEEEauthorhalign}
}
\makeatother % changes the catcode of @ back to 12

% Comment out second line to disable.
\newcommand{\todo}[1]{}
% \renewcommand{\todo}[1]{{\color{red} TODO: {#1}}}

\def\BibTeX{{\rm B\kern-.05em{\sc i\kern-.025em b}\kern-.08em
    T\kern-.1667em\lower.7ex\hbox{E}\kern-.125emX}}

\newcommand{\ignore}[1]{} 
    
\begin{document}

\title{\chameleon{}: Untangling Type Errors Through Interactive
  Visualization and Exploration
}

%\ignore{
\author{\IEEEauthorblockN{Shuai Fu}
\IEEEauthorblockA{
\textit{Faculty of Information Technology} \\
\textit{Monash University}\\
Clayton, Australiay \\
\texttt{shuai.fu@monash.edu}}
\and
\IEEEauthorblockN{Tim Dwyer}
\IEEEauthorblockA{
\textit{Faculty of Information Technology} \\
\textit{Monash University}\\
Clayton, Australiay \\
\texttt{tim.dwyer@monash.edu}}
\and
\IEEEauthorblockN{Peter J. Stuckey}
\IEEEauthorblockA{
\textit{Faculty of Information Technology} \\
%\textit{Data Science and Artificial Intelligence} \\
\textit{Monash University}\\
Clayton, Australia \\
\texttt{peter.stuckey@monash.edu}}
\and
\linebreakand 
\IEEEauthorblockN{Jackson Wain}
\IEEEauthorblockA{\textit{Faculty of Information Technology} \\
\textit{Monash University}\\
Clayton, Australia \\
\texttt{0000-0003-2006-3538}
}
\and
\IEEEauthorblockN{Jesse Linossier}
\IEEEauthorblockA{\textit{Faculty of Information Technology} \\
\textit{Monash University}\\
Clayton, Australia \\
\texttt{0000-0001-6782-7019}
}

}
%}

\maketitle

\begin{abstract}

% \todo{I would start with saying untyped langs popular e.g. Python and the need to address problems this brings...}

Dynamically typed programming languages are popular in education and the software industry. While presenting a low barrier to entry, they suffer from run-time type errors and longer-term problems in code quality and maintainability. Statically typed languages, while showing strength in these aspects, lack in learnability and ease of use. In particular, fixing type errors poses challenges to both novice users and experts. Further, compiler type error messages are presented in a static way that is biased toward the first occurrence of the error in the program code. To help users resolve such type errors we introduce \chameleon{}, a type debugging tool that presents type errors to the user in an unbiased way, allowing them to explore the full context of where the errors could occur. Programmers can interactively verify the steps of reasoning against their intention. Through three studies involving actual programmers, we showed that \chameleon{} is more effective in fixing type errors than traditional text-based error messages. This difference is more significant in harder tasks. Further, programmers actively using \chameleon{}'s interactive features are shown to be more efficient in fixing type errors than passively reading the type error output.
\end{abstract}

\begin{IEEEkeywords}
types, type errors, debugging, visualization, exploration
\end{IEEEkeywords}



\input{sections/introduction.tex}
\section{Chameleon IDE} \label{chameleon}

\chameleon{} comprises two parts: a type inference engine and an original debugging interface. This separate architecture allows us to 1) adapt \chameleon{} into multiple front ends, such as IDE extensions, desktop applications, and web applications, and 2) reuse the debugging interface for different programming languages by replacing the back end type inference engine. The debugging interface is designed from the ground up; the type inference engine is a re-implementation of the original Chameleon with several novel improvements.

\subsection{The Type Inference Engine}
\label{sec:typeinferenceengine}

Chameleon was originally a command-line tool developed in the early 2000s to improve type error reporting and introduced a design for interactive type debugging for the Haskell programming language. Unlike traditional type errors produced by the Glasgow Haskell Compiler (GHC)~\cite{ghc}, which uses a Hindley–Milner type inference system, Chameleon infers types using constraint solving. In Chameleon, Constraints are generated from the source code based on typing rules. In addition, each constraint is labeled with the location (line and column numbers) where it is generated. This set of constraints is consistent if the program is well-typed, inconsistent if otherwise. When a type error occurs, an efficient algorithm is used to derive a minimal subset of the constraints that still contains inconsistencies. This subset is called a Minimal Unsatisfiable Subset (MUS). From this, Chameleon can report a list of locations, using the labels of constraints that are in the MUS. Stuckey et al.~\cite{stuckey2003interactive} showed that program locations linked to the constraints from a MUS are all relevant to the type error and must include the cause of the error. In this paper, we will refer to these locations as \textit{error locations}.

Despite successfully borrowing the underlying ideas, we could not reuse the implementation of Chameleon due to the project language standard and libraries used was out of date. In addition to what Chameleon can do, a few new type inference features were added in the \chameleon{} implementation.



\paragraph{Recovering concrete types from type errors}


Using only constraints from the MUS is sufficient to locating the type error, but to recover types from type errors we need the constraints from parts of the program that irrelevant to the type errors.  For instance, consider an ill-typed 2-tuple where two possible types can be assigned: \texttt{(Int, Int)} and \texttt{(Int, String)}. The types reconstructed from Chameleon may be \texttt{(a, Int)} and \texttt{(a, String)}. Although the recovered types are theoretically correct, they introduce new type variable \texttt{a} making the error message harder to understand.  To solve this issue in \chameleon{}, we first expand the MUS to a maximally satisfiable subset. We do so by removing one constraint from the MUS to get a consistent set, and then repeatedly adding constraints from the global constraint store while maintaining consistency. This process stops when no more constraints can be added. The resulting subset, while not helpful to inform error locations, will produce the most concrete conflicting types. This difference can be seen in figure~\ref{fig:compare-to-original}. 


\begin{figure}[h]
    \centering
    \includegraphics[width=\linewidth]{images/compare-to-original.pdf}
    \caption{
Reporting the same type error, Chameleon uses
\texttt{Int -> a} and \texttt{Char -> a}, while \chameleon{} uses the 
concrete types \texttt{Int -> Bool} and \texttt{Char -> Bool}.
    }
    \label{fig:compare-to-original}
\end{figure}

This has a few benefits. First, concrete types provide extra information to programmers. With a type of \texttt{(Float, Float)}, programmers may want to convey a point in 2d space. However, this information is not preserved in \texttt{(a, Float)}. Second, with concrete types, the debugging front end has more flexibility to change the representation of types. If the front end
understands the type language in Haskell, it is possible to recognize and replace ground facts that are overlapping in both possible types and have the same effect. The other direction is impossible. 

\paragraph{Type error explanation}

In addition, \chameleon{} provides support for type explanation. Similar to the type explanation system ~\cite{jun2002explaining} from Yang,  \chameleon{} is able to produce a human-readable explanation, but for type errors. This is achieved by expanding the metadata fields of each constraint to contain the information needed for type explanation. 

% \begin{lstlisting} [
%     language=Haskell, caption={
%     A simple program that is ill-typed. This program generates two constraints from line 1 and one constraint from line 2.
% }, label={listing:ifelse}
% ]
% if a then b else c
% a = "True"
% \end{lstlisting}

\begin{minted}{haskell}
if a then b else c
a = "True"
\end{minted}

For instance, the program in the listing \ref{listing:ifelse} \chameleon{}  generates the following constraints and labels (in brackets) $T_a = Bool$ (if condition), $T_b = T_c$ (if branches), $T_a= String$  (definition). Clearly, as $T_a$ can not unify with both \textit{Bool} and \textit{String}, this program is not well typed. \chameleon{} can construct a human-readable explanation from the MUS, for example: \texttt{a} has type \texttt{Bool}, because \texttt{a} is the condition of an if statement; however \texttt{a} has type \texttt{String}, because \texttt{a} is defined as the string literal \texttt{"True"}. This low-level explanation is used in a high-level user interface described in section \ref{sub:deduction-steps} to help programmers reason about the validity of error locations.


\begin{figure*}
    \centering
    \includegraphics[width=\textwidth]{images/atonomy.pdf}
    \caption{
        \textbf{The anatomy of \chameleon{}.}
        \chameleon{} consists of two windows: an editor window on the left and a
        debugging window on the right. The editor window is similar to a
        traditional code editor. Fragments of source code may have a highlight
        color (A). Additionally, an explanation layer (B) displays in the editor
        window, if deduction steps are enabled. The debugging window contains
        three panels. A message panel contains an error description (C), and, optionally,
        a list of candidate expression cards (D), a list of deduction steps (E), 
        and a control bar (F) to increment/decrement deduction step. 
        The conflicting types
        panel (G) shows two possible types inferred by the type inference
        engine. A relevant type information panel to show additional information
        that may help understand type errors.
    }
    \label{fig:anatomy}
\end{figure*}


\subsection{The Debugging Interface}

The \chameleon{} debugging interface provides three main features to visualize the relation between an error statement and the error locations in the source code, and to explore the explanation of each error location given by the type inference engine. Further, programmers  can turn on and off features to suit their preferences and debugging needs.



\paragraph{Type compare tool} \label{sub:type-compare}

The type compare tool shows the two conflicting types in different colors, each associated with one or more error locations (Fig. \ref{fig:compare}).  The type error must be fixed by modifying at least one of these locations. The error locations are highlighted with the matching color of the conflicting types. The type compare tool is useful to quickly bisect the type error. If the programmers know the expression's intended type (they usually do), they will be able to eliminate half of the possible locations. To make the bisecting effect more pronounced, a user can hover on one of the possible types and only show the relevant locations that
contribute to that type. This is a convenient way to put the error "under the spotlight".


\begin{figure}[h]
    \centering
    \includegraphics[width=\linewidth]{images/intro-compare.pdf}
    \caption{
        \textbf{\chameleon{} minimal output (with candidate expression cards and deduction step disabled)}. \chameleon{} identified the conflicting types for the function \texttt{a} and associated the relevant locations with each type. In addition, \chameleon{} leaves out the right-hand side of the declaration '\texttt{0}' and '\texttt{1}' because they are unrelated.
}
    \label{fig:compare}
\end{figure}

\begin{figure}
    \centering
    \includegraphics[width=\linewidth]{images/intro-candidate.pdf}
    \caption{
        \textbf{\chameleon{} with only candidate expression cards enabled.}
        \chameleon{} shows that the type error can
        occur in the definition of '\texttt{x}' or '\texttt{y}'.
    }
    \label{fig:expression}
\end{figure}


\begin{figure}
    \centering
    \includegraphics[width=\linewidth]{images/intro-deduction.pdf}
    \caption{
        \textbf{\chameleon{} with both candidate expression cards and deduction steps
        enabled.}
        \chameleon{} explains the type error in four steps. In the screenshot, the active step is step 2, where \chameleon{} shows that the expression \texttt{x} and \texttt{y} should have the same type. 
    }
    \label{fig:deduction}
\end{figure}



\paragraph{Candidate Expression Cards}  \label{sub:candidate-expression}

A candidate expression is an expression that can be inferred to have two conflicting types. 
We propose candidate expression cards to make it clear to the user that there are multiple ways to resolve a type conflict. By contrast,  standard compiler error messages such as those of GHC arbitrarily focus the users attention on only one candidate expression, e.g.\ ``\textit{Couldn't match expected type ‘Char’ with actual type ‘Bool’ in expression x}" here x is the candidate expression. In practice this candidate expression often does not match programmers' intention (Fig. \ref{fig:single-candidate}).  

When a type error is detected, \chameleon{} provides a list of all candidate expressions and programmers are free to choose the problem to resolve by clicking on one candidate expression card. In the example shown in Fig. \ref{fig:expression}, \texttt{x} and \texttt{y} are both candidate expressions. Indeed, fixing one type error will make both expressions well-typed.


\begin{figure}[h]
    \centering
    \includegraphics[width=\linewidth]{images/single-error-location-example.png}
    \caption{
With traditional type error messages, one expression is blamed for causing the type error. This expression may not always match the programmers' intention. 
    }
    \label{fig:single-candidate}
\end{figure}

Selecting a candidate expression card is a way for programmers to communicate to the type checker the knowledge that the other candidate expressions are in fact well-typed. Once a card is selected, the information in the conflicting types panel changes according to the change of candidate expression. In the editor window, some error locations change highlight colors to reflect the updated knowledge of the program. 


This feature can be effective in resolving the problem that traditional tools tend to report errors that happened in source code from third party libraries. With candidate expression cards, a programmer can freely change the context of the type error to the user defined variables and functions to gain a better understanding of why her own code does not match the library code instead of the other direction.


Programmers interact with the candidate expression cards by clicking on them to commit the change of meaning or hovering on them to peek at the effect of the change; the effect is reverted once the cursor moves away.


\paragraph{Deduction steps}  \label{sub:deduction-steps}

Deduction steps are motivated by the lack of explain why the program has  type errors. When compiler reject a program, it dumps the internal state of type checking.  This information may be a result of complex computing but this process is not reported by compilers. For a typical type error shown in Fig. \ref{fig:ghc-error-example}, clearly the evidence for type error are gathered from are generated from the previous two declarations. These have to be re-discovered by the programmers again, using harder and less sound methods. 

\begin{figure}[h]
    \centering
    \includegraphics[width=\linewidth]{images/ghc-error-example.png}
    \caption{
        }
    \label{fig:ghc-error-example}
\end{figure}

Deduction steps allow programmers to explore all the error locations one at a time (Fig. \ref{fig:deduction}). Steps are shown as a list of sequentially  numbered circular buttons (step buttons) and an explanation layer in the editor window. In the explanation layer, two highlighted error locations are outlined, and a line is drawn to connect these two locations to a human-readable text explanation of
their semantic connection. Programmers are free to activate any step. The active step is shown as the green button. When activating a step, some highlights switch color. The message in the explanation layer changes as well. A program in Fig. \ref{fig:deduction} generates a list of steps shown in Fig. \ref{fig:step-interface} left.

Programmers use their mouse and keyboard shortcuts to increment or decrement the step number or jump to any step. Typically, a programmer resolves  type errors by navigating through all the deduction steps and verifying whether each explanation aligns with the intended meaning of the program. Eventually, she will find a step that does not match her intention, and the type error can be fixed by modifying one of the two outlined error locations.


\begin{figure}[h]
    \centering
    \includegraphics[width=\linewidth]{images/step-interface.pdf}
    \caption{
Deduction steps if they are shown all at once (left). In practice, steps are shown one at a time. Programmers increment or decrement the step number using the step control bar (Fig. \ref{fig:anatomy}-F) or by directly clicking on a step button (Fig. \ref{fig:anatomy}-E). To increment or decrement the deduction step can be intuitively thought of as to move the position of the \textit{split point} where the blue and orange highlights divide (right).
        }
    \label{fig:step-interface}
\end{figure}

Internally, deduction steps are different ways to group the error locations into two different highlighted colors. If all the highlights are placed in a list-like data structure, each increment of the step numbers moves the end of one color and the start of the other forward (Fig. \ref{fig:step-interface} right).


\paragraph{Multiple Modes}

Nielson pointed out \cite{nielsen1994usability}  that the two most important issues in designing for usability are understanding the users' tasks and the differences in users. From analyzing how users use \chameleon{}, we realized that the ideal debugging interface should adapt to the specific programmer and programming task. It is common that a programmer wants the debugger to just "show the answer", and it is also possible a programmer wants to dive deeper into the problem domain and search for the optimal solution. To accommodate the need to customize the level of information density and granularity of control, \chameleon{} provides three modes: basic, balanced, and advanced. Programmers can switch between modes by clicking on the mode switching toggles (Fig. \ref{fig:anatomy}-I). The features accessible from different modes are shown in table~\ref{tab:chameleon-features}.

\begin{table}
    \centering
    \begin{tabular}{ l l  }
     Mode & Features \\ \hline
     Basic Mode & Type Compare Tool \\ \hline
     Balanced mode & Type Compare Tool \\
     & Candidate Expression Card \\  \hline
     Advanced mode & Type Compare Tool \\
     & Candidate Expression Card \\
     & Deduction Steps \\
    \end{tabular}
    \caption{\chameleon{} modes and features}
    \label {tab:chameleon-features}
\end{table}



\input{sections/walkthrough.tex}
\section{Evaluation}

Over 12 months, several studies were conducted to answer new research questions or rephrase existing ones. These studies are grouped in 3 phases, each tested a different version of \chameleon{} with human participants. The number of participants and their haskell experience is shown in the table below. 

\begin{figure}[h]
    \centering
    \includegraphics[width=\linewidth]{images/participant-demographic.pdf}
    \caption{}
    \label{fig:participant-experience}
\end{figure}

A summary of the timeline of the user studies, their major questions and conclusions is given in Figure~\ref{fig:timeline}.

\begin{figure}[h]
    \centering
    \includegraphics[width=\linewidth]{images/timeline.pdf}
    \caption{The timeline of \chameleon{}  development and 
    evaluation.}
    \label{fig:timeline}
\end{figure}

\subsection{Experiment Design}
\subsubsection*{\textbf{Recruitment}}

Participant recruiting channel was the social news aggregation website Reddit. Specifically, our user study advertisements were posted in the Haskell programming language community "r/haskell" and a general programming language-focused community "r/programminglanguages". Recruiting from social media allowed us to access a more diverse demographic that better represent the true population of Haskell programmers. The participation is fully anonymized. The detailed ethical implications of these experiments are reviewed and approved by the IRB of the institution of the authors.

One consequence of our recruiting approach is it is harder for previous participants to enter an later study. To minimise the lurking variable of previous experience, we use new code challenge every study and conduct trial runs in every study to bring new participants up to speed.

\subsubsection*{\textbf{Experiment setting}}
The experiment took place remotely and unsupervised. Participants took the study online via web browser and at the physical venue of their choosing. All user studies use a web-based debugging environment developed by the authors. Conducting the studies online helped us avoid variation when performing tasks in unfamiliar places and using different setups. The downside is that to intervene when a user encounters bugs or usability issues mid-study is impossible. To ensure there is no major usability issue that could lower the quality of data collection, we conducted cognitive walkthroughs and sandbox pilots before running each study.

\begin{figure}[h]
    \centering
    \includegraphics[width=\linewidth]{images/procedure.pdf}
    \caption{}
    \label{fig:study-process}
\end{figure}

\subsubsection*{\textbf{Training and group assignment}}
After consent, participants received interactive training on every function of the tool interface and interactive features. Participants were also shown the important functionalities of the tool interface in a summarized cheat sheet. Participants had access to the cheat sheet at all times during the study. Participants were given 4 trial runs (2 for each setting) before the data collection starts. 

All the user studies used a within-subject design to evaluate the effectiveness of different tools or feature sets while counterbalancing the difference in programming proficiency between participants. In each user study, participants were required to complete a series of programming tasks (8 for user studies 1 - 3, 9 for user study 4). At each task, a participant receives a single Haskell file that contains one or more type errors. The participant was asked to make the code type check with the help of the tool given for this task.


% During each study, a participant received two (user studies 1 - 3) or three (user study 4) different debugging tools. The debugging tools were different systems (user studies 1 - 3) or different feature sets of the same system (user study 4). For each participant, the debugging tools alternative through consecutive tasks to ensure the fatigue level does not impair the rigorousness of data collection.

\subsubsection*{\textbf{Measurement}}
Time is measured from the start of each task to the first time the program is successfully type-checked and also passes all the functional tests. The data is automatically recorded by the online debugging environment. To not introduce a barrier to completing the study, every task can be skipped if the participant made three attempts or spent over 1 minute on the task.


After completing all the tasks (8 in user studies 1-3, 9 in user study 4), users are prompted to fill in a debriefing survey. The survey questions include their Haskell programming experience and feedback on the tools and feature sets participants used during the study.


We used an browser session recording tool~\cite{openreplay} to record the user study sessions. These recordings are used to find usability issues in the study and recognize general patterns. However, the recording technology does not provide high enough sampling rate for us to be confidently used for rigorous analysis.

\subsection{\chameleon{} User Studies}


\subsubsection{\textbf{Version 1}}  
\chameleon{} version 1 provides the rewrite of the Chameleon type inference engine and the type compare tool to show the alternative types and possible error locations in matching colors.

\begin{figure}[h]
    \centering
    \includegraphics[width=\linewidth]{images/chameleon-v1.pdf}
    \caption{
        \chameleon{} Version 1
        }
    \label{fig:chameleon-v1}
\end{figure}

Two user studies were conducted to test \chameleon{} version 1, to compare the effectiveness of solving type errors using \chameleon{} and GHC compiler error messages. We choose GHC compiler error messages as the baseline because it is the canonical tool for working with type error in Haskell. Although high-level tools like Haskell Language Server exist, they generally relay the GHC error messages verbatim. 


Eight tasks were given in both studies. In the first study, the tasks were taken from a exercises from the exercises of the Haskell programming units. These tasks cover a variety of type errors. \tf{break down the type error classes.} However these tasks, as shown from the result and feedback of user study 1, were too simple to fully evaluate the two options. In the second study, the tasks are sourced from the top 20 Haskell topics on GitHub~\cite{githubHaskell}. The authors manually added type errors. 


% The study investigates the effectiveness of \chameleon{} compared to the GHC compiler error messages. We chose GHC compiler error messages as it is the canonical tool for debugging type errors in Haskell. Although high-level tools like Haskell Language Server exist, they relay the GHC error message verbatim for type errors. Participants are asked to complete 8 tasks. The tool participants used during each task alternated between \chameleon{} and GHC. Task programs were sourced from HaskellWiki \cite{haskellwiki}. The author manually added type errors. The errors cover a range of common Haskell type errors, including abstract data types, wrong arity, control expressions (if and case), infinite types, and tuples. The lines of code (LoC) range from 7 to 17 (mean = 11, median=10.5).


% In total 39 participants finished the study. Among them, 12 participants have over five years of Haskell experience, and 5 participants have three or four years of Haskell experience. And 8 participants have one or two years of experience, and 5 participants used Haskell for under a year. The rest left the question unanswered.

\begin{figure}[h]
    \centering
    \includegraphics[width=\linewidth]{images/r1-data.pdf}
    \caption{Time to complete by task in user study 1 with confidence interval}
    \label{fig:r1-analysis}
\end{figure}


\begin{figure}[h]
    \centering
    \includegraphics[width=\linewidth]{images/r2-data.pdf}
    \caption{Time to complete by task in user study 2 confidence interval}
    \label{fig:r2-analysis}
\end{figure}

\subsubsection*{\textbf {Results}}

From the data collected during user study 1 (\ref{fig:r1-analysis}), we could not identify which group performed better (p-value =0.2041, Wilcoxin signed test). For the trivial challenges we set for users, the individual differences are generally more significant than differences between treatments. 
The most common feedback (7 out of 35) from the user study 1 was that the tasks were too trivial to invite meaningful evaluation. One participant put it, "Looks nicer than GHC, but without trying it on something more complicated, I cannot conclude whether it would help me in practice".

One interesting exception is task 8, where the \chameleon{} group outperformed the GHC group more consistently and with a larger effect size (p-value = 0.002456, Wilcoxin signed test). The discrepancy is thought to be related to the nature of Task 8:
\begin{itemize}
    \item {It has a longer source file than other tasks (only shorter than task 6, but task
    6 contains two independent type errors while task 8 is one connected task
    error);}
    \item {It is more complex than others (involves abstract data types and function application); and
    }
    \item {
        GHC struggles to produce a relevant error message for this type of error.
    }
  \end{itemize}
It can be further shown by examining the process some participants took to solve the problem.

  

% \begin{figure}[H]
%     \centering
%     \Description{A screenshot from user study 1}
%     \includegraphics[width=\textwidth]{round1-screenshot.pdf}
%     \caption{
% One participant working with \chameleon{}  managed to identify the most probable location of the error (the application of `fromMaybe'` on line 11) by hovering over the two alternative types. The participant then quickly realized that the first argument `0` (highlighted in orange) is inconsistent with the definition where it is case matched to a `Nothing` value (highlighted in purple). After two retries with short hesitation the participant found the correct fix (by reversing the order of `0` and `n`).
%     }
%     \label{fig:r1-task8}
% \end{figure}


% \begin{figure}[htb]
%     \centering
%     \Description{A screenshot from user study 1}
%     \includegraphics[width=\textwidth]{round1-screenshot-ghc.pdf}
%     \caption{
%         In task 8, participants using GHC took longer pauses at this task before committing to further investigation, either carefully reading the whole text or figuring out the meaning of the error message. The GHC error message is unfortunately unhelpful for this task.
%     }
%     \label{fig:task8-ghc}
% \end{figure}

In comparison, the \chameleon{} group solved the type error faster than the GHC group in almost all tasks (p =0.00742, Wilcoxin signed test) in study 2 (figure \ref{fig:r2-analysis}), barring task 1. The authors are unclear about the cause of low performance from \chameleon{} in task 1. It is suspected that some participants spent more time exploring the interface of \chameleon{} due to its unfamiliarity. From the video recordings, we saw many \chameleon{} users confidently skip reading unrelated chunks of code, while GHC users generally read through the whole program.


In harder problems and messier code, we notice programmers start to report the benefits of \chameleon{}. "Its most useful feature that I noticed was that it points out the locations of both conflicting uses; GHC often makes it difficult to figure out how it's coming to a conclusion about a type." reported one participant. "I think \chameleon{}  does a much better job than GHC's error messages. I like that it shows the sources for the type judgements. This makes it quite easy to figure out how to rectify errors." reported another participant.



\subsubsection{\textbf{Version 2}}  \label{sub:us4}

Version 2 is latest version of \chameleon{}. Not only the added features the deduction steps and candidate expressions are We designed an exploratory study to get better understanding of how users learn and use the multi-mode design of \chameleon{} version 3. During the study the initial mode of each task alternative through the three different modes, and repeat three cycles in nine tasks. The order of the three modes in each cycle is counterbalanced among all participants. However, participants are free to switch to other modes at any time. 



\subsubsection*{\textbf{Results}} 

The study of version 3 \chameleon{} is exploratory, in that we were hoping to let programmers discover their own way of using the tool. In post-hoc analysis of the collected log data, we were able to extrapolate some interesting patterns of how the tool was used. We share a few of our observations in this section.


The most striking feature of the data is users' tendency of using tools differ wildly. That is to say, some users used the feature extensively however some users completed the tasks without actively explore the given information. Based on this difference we divided the users into three groups.

\begin{tabularx}{\linewidth}{ 
  | >{\raggedright\arraybackslash}X 
  | >{\raggedright\arraybackslash}X  | }

    \hline
        Interaction level & Description \\ \hline
        Minimal Interaction & Users completed the tasks by making changes in source code, type checking, and reading error messages. \\ \hline
        Low Interaction & Users only actively used universal features in all modes, for example, hovering on "Possible type 1" and "Possible type 2" to narrow down error space. \\ \hline
        High Interaction & Users did everything from the Common feature group but used features specific to Balanced and Advanced mode, such as activating steps and expression cards. \\ \hline

\end{tabularx}



As shown in  figure \ref{fig:r4-analysis}, the time to complete each task roughly relates to the interaction level of participants. Participants with higher interaction levels generally performed better in the tasks, and the lowest level was worse (one-way ANOVA since we have three samples, p = 0.0368). The results from three tasks stand out from the general trend: in Tasks 4 and 6, higher interaction users performed worse, and in task 9, the general trend is exaggerated. As discussed in user studies 1 and 2, it is likely related to the nature of these tasks. Tasks 4 and 6 are shorter tasks in the study. The ideal fixes for these two tasks are placed relatively early in the source code (both in the first two lines of the source code). These may be useful because following the natural reading order will allow participants to skip lines on the bottom, and thus the advantage of \chameleon{} is less important. On the other hand, task 9 is the longest task of all. It also involves harder problems such as mutually recursive type definitions.

\begin{figure}[h]
    \centering
    \includegraphics[width=\linewidth]{images/r4-analysis.pdf}
    \caption{Time to complete by task in user study 4 with confidence interval}
    \label{fig:r4-analysis}
\end{figure}


\begin{figure}[h]
    \centering
    \includegraphics[width=\linewidth]{images/r4-task9.png}
    \caption{
High interaction users were able to locate the exact line for a potential fix after navigating to deduction step 6 or 7, which directly revealed the true cause of the type error. After this, 12 out of 15 high interaction users provided the most ideal fix in the first try. Low interaction group  takes longer time to examine the problem, provided wrong fixes in the initial trials. One participant fixed a minor run-time issue (a problematic pattern matching of `[attributes]` on line 18) however failed to identify the culprit of the type error.
    }
    \label{fig:r4-task9}
\end{figure}


Another observation is when using the mode switching feature of \chameleon{}, programmers generally switching form a less informative display to a more informative one (Fig. \ref{fig:r4-mode-switching}). 

\begin{figure}[h]
    \centering
    \includegraphics[width=\linewidth]{images/r4-mode-switching.png}
    \caption{
        The rows represent the finishing mode, columns starting modes.
    }
    \label{fig:r4-mode-switching}
\end{figure}



\section{Discussion}

\todo{Suggest try and strucuture this a bit more e.g. each para bold the key lesson @ start of para. Can you group into implications for research/implications for practice to make more actionable??}

This paper presents the interactive type debugging tool \chameleon{} and charts the evolution of its design across several iterations in response to user evaluation and feedback, as well as examining the effectiveness of the general approach compared to traditional static type error messages. We found that programmers using \chameleon{} type compare tool are able to debug errors faster than using traditional text-based error messages. This effect is shown more clearly when the task is not trivial. We found that programmers who actively use ChameleonIDE's interactive features (candidate expression cards and deduction steps) are more efficient in fixing type errors than simply reading the type error output. In this section, we will discuss a few interpretations of the results.


From the results of user studies 1 and 2, we observed that the choice of debugging tool had little effect on how fast programmers solve type errors. Conversely, when facing more realistic problems (longer source code, error locations more scattered), programmers are more effective using \chameleon{}. We believe one factor of this effectiveness of \chameleon{} is to reduce the amount of reading time. Jbara \cite{jbara_how_2015} showed that reading source code is generally the initial step of solving programming problems and is done in several passes. Although traditional compiler error message tools initially show fewer locations, it is unreliable, meaning that programmers have to expand the reading span without clear guidance. In contrast, Chameleon shows more error locations initially. However, the completeness of error locations assures programmers which part of the source code can be safely skipped.


From the results of user studies 3 and 4, we found that programmers who use interactive tool fix type errors faster than the ones who passively read the error output. This difference is stronger in harder tasks. We speculate that one factor of this result is that  \chameleon{} helps to develop debugging plans. We observed that when working with \chameleon{}, programmers form different debugging tactics to attack the problem. Among the high interactivity participants in user study 4, some programmers cycle through deduction steps as a guide to reading source code; some programmers navigate to both ends of the deduction steps where types are normally grounded and concrete. In contrast, minimal interactive participants generally form more similar plans, including carefully reading the program text and manually annotating expressions based on their understanding of the program.

\begin{listing}[!ht]
\begin{minted}{haskell}
f z     
    | z == 3 = False
    | z == '4' = True
\end{minted}
\caption{     In this simple program, \chameleon{} report the error happened in \texttt{f} and \texttt{z}.}
\label{listing:2}
\end{listing}

We speculate another factor of the effectiveness \chameleon{} interactive debugging tools is they help programmers effectively chunk intermediate information. With the program showing in the listing \ref{listing:2}, \chameleon{} offers two candidate expressions: \texttt{f} can be typed as \texttt{Int -> Bool} or \texttt{Char -> Bool}; \texttt{z} can be typed as \texttt{Int} or \texttt{Char}. Although, in theory, these two statements are to the same effect, programmers are often required to compute the latter from the former or vice versa. And this computation may carry out multiple layers. Programmers have to remember all the intermediate types and their reasoning throughout such mental gymnastics. Assisted by candidate expression cards and deduction steps, this intermediate information is externalized on screen and can be retrieved anytime. A recent study on working memory \cite{crichton_role_2021} suggested this approach may provide a positive effect in helping programmers manage cognitive load and free up working-memory space for high-level thinking.


In our studies, \chameleon{} shows a few different designs for type error visualization and interaction. However, we are far from exhausting the search space. One challenge we notice in the current design is that programmers have to shift their focus between the editor on the left and the \chameleon{} debugging interface on the right. Using the hover popup window in most mainstream IDEs may reduce the context switching during debugging.

On a similar note, the current implementation of \chameleon{} requires non-trivial adaptation for editors such as VS Code and IntelliJ due to the overlay explanation layer. This type of error visualization is non-standard in mainstream editors. However, there may exist alternative representations of \chameleon{} error reporting using only the features available in major editors and IDEs. It is especially beneficial to represent \chameleon{} errors using a universal debugging middle layer such as language server protocol (LSP). This will allow \chameleon{}  to be adapted into various coding environments which support an LSP back end.

Another area for future study is to extend the evaluation process. In studies 1b and 2, we used harder tasks compared to study 1a. However, the difficulty level is nowhere as hard as a professional Haskell programmer may face in the production Haskell codebase. It would be interesting to see how \chameleon{}  faces type errors that span multiple files and packages. Another dimension of difficulty is higher-level extractions. Haskell programs get more tricky to the debugger when more abstractions are added to the codebase, such as Monads, Monad transformers, and Lenses. These topics were not included in our study to accommodate novice users. However, observing how expert Haskell programmers use \chameleon{} features to solve advanced problems will gain valuable insight into the true usability of the tool.

Another direction to expand the work is in data collection. The online study limited the amount of qualitative data we could gather from the users. Although we handed out debriefing surveys after each study, it is no substitute for more in-depth qualitative study methods. Inviting users for speak-aloud walkthroughs and interviews are viable methods to unveil deeper usability issues about \chameleon{}.

Our study focuses on the Haskell language for its popularity in the academic world. However, the low quality of textual type errors is a problem not limited to Haskell. Modern statically typed languages more or less all share the same problem. We believe the underlying type reporting technique and algorithms can be generalized to other languages. It will be exciting to see how interactive debugging features perform in other paradigms, such as  imperative languages and incremental type systems.

Our design of \chameleon{}  emphasizes allowing programmers to update the type error based on their prior knowledge. It unfolds to answer three questions: which expression should be considered un-typeable? Between multiple possible types, which one should be considered intended? Among all the possible locations, which one should be considered the cause? \chameleon{} answers all three questions by leaving it to the programmers to decide. However, it is possible and likely that some programmers enjoy conclusive error messages instead of flexible ones provided by \chameleon{}. Using heuristic methods may achieve the best of both worlds. It may produce effective results by guessing the possible un-typable expression based on the position it appears in the syntax tree and project structure or guessing the intended type signature by comparing the number of supporting slices in the source code.

\subsection{Threats to validity}

\todo{Put threats BEFORE Discussion and at end Eval section?}

\todo{These are too terse/unclear - need expanding. Suggest classify as internal / external threats etc .  What about background and number of participants as threats??}

One consequence of our recruiting approach is it is harder for previous participants to enter a later study. To minimize the lurking variable of previous experience, we use new code challenges in every study and conduct trial runs in every study to bring new participants up to speed.

Conducting studies remotely and unsupervised is that intervening when users encounter usability issues mid-study is impossible. To ensure there is no major usability issue that could lower the quality of data collection, we conducted cognitive walkthroughs and sandbox pilots before running each study.
\section{Related Work}
%%%  Todo: Discussing "Practical SMT-based Type Error Localization"
Researchers have long realized that incomprehensible type errors result as a consequence of standard approaches to the type inference process. Many studied [Citation here] the approach of finding all locations that contribute to a type error. Haack and Wells~\cite{haack_type_2004} noted that ``\textit{Identifying only one node or subtree of the program as the error location makes it difficult for programmers to understand type errors. To choose the correct place to fix a type error, the programmer must find all the other program points that participate in the error.}''  Type error slicing~\cite{haack_type_2004} is a technique that finds locations that are complete and minimal for the type error. Internally labeled constraints and Minimal Unsatisfiable Set (MUS) manipulation are used to generate these slices. The language supported in Haack's work was a subset of Standard ML. The original Chameleon~\cite{stuckey_interactive_2003} used a similar technique but generate constraints in Constraint handling rules (CHR). As a result,  Chameleon supported advanced type-level features (type classes and functionally dependent types). The project also introduced the ability to query type information through a command line-style interface. Although Chameleon was firmly grounded in results from type theory, its designs were never evaluated with user studies.

One direction to approach type error debugging is by suggesting solutions to fix the type error. Sheng Chen's works \cite{chen_counter-factual_2014, chen_efficient_2020, chen_improving_2022} on counterfactual typing are able to suggest changes to make the ill-typed program type-check. While this is a promising ground, the work is not evaluated with human participants. \cite{lerner_searching_2007} studied an approach of repeatedly modifying the AST (abstract syntax tree) and querying the compiler/type checker to know whether the new program is well-typed. This approach allows the algorithm to suggest syntax changes that could fix the type error. Benefiting from not needing a type system implementation and being language agnostic, the underlying idea was implemented and evaluated in two different languages with ten university students. While suggesting syntax changes is very intuitive in practice, this approach has limitations for teaching and learning purposes, as it is not able to explain why the type errors occur and why fixes are suggested. The lack of explanation is due to the inquiry into the underlying type-check being a binary state. Deep type-level analysis (typing rules and language features) was not communicated to the inference algorithm.

Another related approach is through ECEM (Enhanced compiler error message). Through a series of mixed-method studies, Prather showed \cite{prather_novices_2017} that ECEM has a positive result in understanding compiler errors. Decafe \cite{becker_effective_2016} is a tool that can rephrase the Java compiler error messages into their enhanced version. In a study of over 200 CS1 students, Decaf was shown to reduce overall errors in their coding practices. Berik proposed a framework \cite{barik_how_2018} for constructing compiler error messages based on argumentation theory. In this proposal,  Berik argued that error messages following a simple argumentation layout or an extended argumentation layout are more human-friendly.  These works show the significance of improving the language in the compiler error messages. Most principles and suggestions are followed in \chameleon{} in constructing error statements. However, they were not targeting type errors alone but compiler errors (some even include run time errors) in general. The nuances of type errors, such as alternative typing, were not considered. 

% \begin{figure}[h]
%     \centering
%     \includegraphics[width=\linewidth]{images/original-chameleon.pdf}
%     \caption{
% Originally, Chameleon was a command-line tool that can list all the potential causes of a type error and display the two branches in two forms of highlight (right), achieved with ASCII colors. In addition, it allows a few commands (debug and explain) in the debugging shell to help debug type errors (left).}
%     \label{fig:original-chameleon}
% \end{figure}


% The ChameleonIDE error reporting The error message of ChameleonIDE closely
% follows the argumentation structure outlined by Titus. Basic mode represent a
% simple structure argument, where the claim "The expression e can have two
% conflicting types" and the possible type 1 and two serves as grounds. When
% inspecting each possible types, the type judgements are treated as claim, and it
% is supported by the grounds: "Inferred from the orange/blue highlights on the
% left side". On the other hand, the similar type error message in GHC "Couldn't
% match expect type T1 with actual type T2" is "a ground masquerading as a clam",
% Barik commented. The balanced mode and advanced mode serve as a way to elaborate
% arguments from simple arguments into extended arguments.

% Another related topic is type-explanation~\cite{yang1999explaining, jun2002explaining}. Yang~\cite{jun2002explaining} showed an alternative type inference system capable of producing a human-like text explanation for why expressions are assigned certain types. A good explanation is drawn from surveying how human experts explain types. The resulting algorithm $\mathcal{H}$, generates a succinct explanation of the type inference steps to avoid using internal constructs (such as type variables). The explanation system has the advantage of acting like a human expert. However, when presented as text-based output, explanation systems have the potential to become verbose when types are complex or variable names are long. In \chameleon{}, we attempt to address this problem by showing one step of explanation at a time and referring to variables instead of spelling out the full name.


Debugging using a GUI Interactive Development Environment has been the standard practice for a long time. Compared to a command line-based interface, a graphic user interface provides programming tools that can show information hierarchy, provide immediate feedback to changes, and display complex visualizations. Hazel Tutor \cite{potter_hazel_2020} is an interactive type-driven environment for the OCaml language. It can automatically fill type holes by suggesting template expressions (strategies as called by the authors) through a popup window. It also provides a cursor-based type inspector that allows programmers to query the types of different parts of the program. Whyline~\cite{ko_finding_2009} is a Java debugging system that allows a user to ask questions like "why does variable X have value Y." It also allows users to interactively ask follow-up questions to gain further knowledge of the nature of an error.  These debugging tools are important motivations for developing \chameleon{}. However, they focus on different contexts of the debugging process. Java Whyline mainly tackles the problem of unintended run-time behavior, while Hazel Tutor specializes in interactive actions supporting type holes.



\section{Conclusion}

We present \chameleon{}, a type debugging tool for the Haskell programming language. Its constraint-based type inference engine provides accurate and comprehensive type error reports. \chameleon{} also features an interactive debugging interface with three different debugging modes, enabling programmers to navigate deep type relations and triangulate conflicting statements in source code. Our studies evaluated the tool's design with programmers. We found that, particularly for more complex tasks, \chameleon{} helped programmers to fix type errors more quickly than traditional text-based error messages.  Further, programmers actively using \chameleon{} interactive features are shown to fix type errors faster than simply reading the type error output.



\bibliographystyle{IEEEtran}
\bibliography{ICPC}


\end{document}
