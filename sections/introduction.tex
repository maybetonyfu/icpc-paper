Compared to dynamic type systems, static type systems offer programmers opportunities to weed out a large number of errors at compile time (reducing the need for run-time debugging) and to increase usability~\cite{mayer2012static} and codebase maintainability~\cite{kleinschmager2012static}. In recent years, the programming community has seen a trend of migrating away from dynamically typed codebases due to the lack of maintainability and refactoring safety when systems reach large scale and complexity~\cite{chatley2019next700}. These migration methods include the introduction of gradual type annotations (e.g., JavaScript to TypeScript or Python with mypy) or transition to a modern strongly-typed language (Scala, Rust, Haskell, PureScript, etc.). However, with dynamically typed languages (e.g., JavaScript, PHP, and Python) still being most peoples' formative experience with programming and with such languages now firmly entrenched in education~\cite{so2022survey}, this transition may pose difficulties for programmers who lack experience writing type-safe programs.


Programmers find challenges adopting typed languages in their practice. They tend to be verbose, hard to learn, and generate bad error messages. 
One obstacle programmers face when using a statically typed language is understanding and resolving type errors~\cite{marceau2011measuring, tirronen2015understanding}, especially when dealing with modern type features such as polymorphic types and implicit typing. Studies show most statically typed languages produce unhelpful and even misleading type errors~\cite{yang2000improved, some more proof}. Symptoms of bad error messages include cryptic language exposing internal constructs of the compiler and incomplete or wrong error locations. These usability issues poses challenges to learners and teachers of the languages. impede the wide adoption of statically typed languages, especially for programmers who are now accustomed to the mindset of writing un-typed programs.


This paper introduces \chameleon{}, an interactive debugging tool to visualize critical context of a type error (where it happened or could have happened, and which chunks of code caused it) and allow programmers to interactively explore all the parts of code where multiple types can be inferred and to resolve that ambiguity. Most noticeably, features (Section \ref{sub:type-compare}) provided by \chameleon{} are the type compare tool, the deduction step feature , and the candidate expression card feature.  In addition, we integrated all the features into a unified debugging interface and provided functionality for programmers to enable and disable the features based on their preferences and debugging needs. \chameleon{} is open-source and is available in both web~\cite{chameleon} and desktop versions. The current implementation of \chameleon{} targets the Haskell 2010 language standard. However, it is planned to extend it to support other programming languages using the same underlying ideas and techniques. Historically, Haskell has been a test bed for advanced language features, and it is common for features established in Haskell to be transferred to mainstream programming languages. Type classes, an implementation of generic programming, were introduced to Haskell in 1988~\cite{hudak2007history}, and now it can be found in most popular languages such as C\#~\cite{csharpgenerics}, Java~\cite{javagenerics}, and TypeScript~\cite{tsgenerics}.

% \chameleon{} uses underlying type inference algorithms adapted from the command-line tool, Chameleon, developed in 2005 \cite{chameleon}. As described in Section \ref{sec:typeinferenceengine}, Chameleon computes the Minimum Unsatisfiable Subset (MUS) \tf{Explain what mus is} of type constraints in order to identify a set of code locations where multiple types can be inferred. Compared to standard compiler messages, which arbitrarily report the first place a type conflict arises, Chameleon error messages give a lot more context to the programmer to help them correctly resolve the conflict to match the program's intent. However, the Chameleon system was never tested with users.  Therefore, the first contribution of this paper is to address the research question: \textit{Is a minimally interactive version of Chameleon's multipart type-conflict display more effective in supporting type error debugging than traditional error messages? } (Sections~\ref{sub:us1} and \ref{sub:us2}).

% After affirming that the information Chameleon provides is beneficial for more complex type errors, we ask \textit{Do programmers benefit from the interactive exploration of type error locations?} (Sections~\ref{sub:us3} and \ref{sub:us4}). We then explore \textit{How do programmers use the interactive exploration features to solve type errors?}  (Section~\ref{sub:us4}).


This paper makes the following contributions:
\begin{itemize}
\item { We provide the design and implementation of the \chameleon{}, to visualize crucial context of a type error and allow programmers to explore and verify the error locations in small chunks interactively.  }
\item {
    We report the result of four experiments designed to evaluate \chameleon{}.}
\end{itemize}

Our experiments showed that programmers using \chameleon{} fix type errors faster than with traditional text-based error messages. This difference is more significant when solving harder tasks. Further, programmers who actively use \chameleon{} interactive features fix type errors faster than simply reading the type error output.
